


Um von erweiterten topologischen Quantenfeldtheorien (ETQFT) reden zu können,
muss man natürlich zunächst wissen, was topologische Quantenfeldtheorien (TQFT) sind.

In der allgemeinen Relativitätstheorie wird die Raumzeit als Mannigfaltigkeit dargestellt.
Sei nun $Z(M)$ die Überlagerung der quantenmechanischen Zustände auf der
$d$-dimensionalen Mannigfaltigkeit $M$.

% Darstellung der TQFTs als Funktoren von der Kategorie der Mannigfaltigkeiten in die Kategorie der C-VR
% Isomorphismen sind wohl die Diffeomorphismen,was sind aber die nicht-Isomorphismen?
% TODO



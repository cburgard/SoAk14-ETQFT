


%Um von erweiterten topologischen Quantenfeldtheorien (ETQFT) reden zu können,
%muss man natürlich zunächst wissen, was topologische Quantenfeldtheorien (TQFT) sind.

In der allgemeinen Relativitätstheorie wird die Raumzeit als Mannigfaltigkeit dargestellt.
Insbesondere ist der Raum zu jeder Zeit als Mannigfaltigkeit darstellbar. Wir betrachten vereinfachend
den Raum als eindimensionale Mannigfaltigkeit ohne Rand oder noch weiter vereinfachend als Kreis oder (endliche)
Menge von Kreisen. Die Kreise kann man in einer Ebene als nebeneinanderliegend betrachten. Da sich der Raum in der
allgemeinen Relativitätstheorie über die Zeit verändern kann, haben wir im Allgemeinen zu verschiedenen Zeitpunkten
verschiedene Mannigfaltigkeiten die den Raum darstellen. Nun möchte man von der Mannigfaltigkeit $M_1$ zu einem Zeitpunkt
kontinuierlich zur Mannigfaltigkeit $M_2$ zu einem anderen Zeitpunkt kommen.
Legt man nun die Ebenen in denen sich $M_1$ und $M_2$ befinden parallel in einen Raum und stellt sich eine Fläche vor die
komplett zwischen diesen Ebenen liegt und als Ränder genau $M_1$ und $M_2$ hat, so hat man wenn man auf dieser Fläche
von der einen Ebene zur anderen läuft einen kontinuierlichen Übergang zwischen den beiden Mannigfaltigkeiten gefunden.
Im einfachsten Fall sind $M_1$ und $M_2$ einzelne Kreise, dann wäre eine solche Übergangsfläche beispielsweise ein Rohr.
Nach der intuitiven Einführung der 2-dimensionalen Kobordismen die allgemeine Definition:

\begin{Def}
  Die Kategorie $\category{Cob}(n)$ der $n$-dimensionalen Kobordismen besteht aus:
  \begin{itemize}
    \item Ein Objekt von $\category{Cob}(n)$ ist eine geschlossene orientierte $(n-1)$-dimensionale Mannigfaltigkeit.
    \item Für ein Paar von Objekten $M,N \in \category{Cob}(n)$ ist ein Morphismus von $M$ nach $N$ ein Bordismus:
    eine orientierte $n$-dimensionale Mannigfaltigkeit mit Rändern $M$ und $N$. Zwei Mannigfaltigkeiten $B$ und $B'$ stellen den gleichen Morhismus dar,
    falls ein orientierungserhaltender Diffeomorphismus zwischen ihnen existiert, der mit den Rändern verträglich ist.
    \item Für jedes Objekt $M \in \category{Cob}(n)$ stellt $M \times [0,1]$ die Identität dar.
    \item Die Komposition von Morphismen ist durch das Verkleben der entsprechenden Bordismen gegeben.
  \end{itemize}
\end{Def}

Sei nun $Z(M)$ der Raum der quantenmechanischen Zustände auf der Mannigfaltigkeit $M$ zu einem Zeitpunkt. Zu einem späteren Zeitpunkt, sei $Z(N)$ der entsprechende
Raum für die Mannigfaltigkeit $N$. Wenn man nun $M$ und $N$ und einen Morphismus in $\category{Cob}(n)$ von $M$ nach $N$ hat, so möche man natürlich berechnen
wie sich der quantenmechanische Zustand von $M$ nach $N$ weiterentwickelt, also ein Morphismus von $Z(M)$ nach $Z(N)$. Die TQFTs leisten genau das, sie stellen also
einen Funktor von der Kategorie der Kobordismen $\category{Cob}(n)$ in die Katergorie der Vektorräume $\category{Vekt}_k$ dar, in welchen die quantenmechanischen
Zustände dargestellt werden.

\begin{Def}
  Eine topologische Quantenfeldtheorie ist ein (symmetrisch monoidaler) Funktor $Z: \category{Vekt}_k \rightarrow \category{Vekt}_k$ mit folgenden Daten:
  \begin{itemize}
    \item Für jede geschlossene $(n-1)$-dimensionale Mannigfaltigkeit ein Vektorraum $Z(M)$.
    \item Für jeden orientierten Bordismus $B$ von $M$ nach $N$ eine lineare Abbildung von Vektorräumen: $Z(B): Z(M) \rightarrow Z(N)$
    \item eine Menge von Isomorphismen $Z(\empty) \cong k$, $Z(M\sqcup N) \cong Z(M) \otimes Z(N)$.
  \end{itemize}
\end{Def}


